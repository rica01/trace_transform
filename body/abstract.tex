% Now let put the abstract in two languages
\doublespacing

\vspace*{\fill}
\begin{center} 
    \textbf{Resumen}
\end{center}

Las imágenes aéreas o satélites se puede utilizar para producir mapas de cobertura de terreno, que son un herramienta útil e importante para tomadores de decisiones en varios campos de trabajo, incluyendo biodiversidad, telecomunicaciones y gestión de desastres naturales entre otros.

El método de la Transformada de Trazo puede usarse para procesar estas imágenes. Este método extrae características de las imágenes al aplicar una serie de funcionales de manera sucesiva para producir una representación numérica que se evalúa en la clasificación más adelante. Este modelo depende de varios factores para alcanzar una operación eficiente, entre ellos, los parámetros de frecuencia de las trazas, el tipo de clasificador utilizado y los tipos de cobertura de terreno a evaluar.

Es el propósito de esta investigación producir un modelo de sensibilidad a estos parámetros para la clasificación de texturas en imágenes aéreas utilizando la Transformada de Trazo.

La experimentación sobre el tiempo de extracción y precisión de clasificación revelo que los parámetros de frecuencia, en especial el parámetro de selección de píxeles, $\Delta \tau$ y el clasificador utilizado tiene un gran impacto sobre ambas medidas.

\textbf{Palabras clave}: Transformada de trazo, característica triple, procesamiento de imágenes, tiempo de extracción, precisión de clasificación, imágenes aéreas.
\vspace*{\fill}


\newpage


\vspace*{\fill}

\begin{center}
    \textbf{Abstract}
\end{center}


Aerial or satellital images can be used to produce terrain coverage maps, which in time are a very useful and important tool for decision-making people in several fields including biodiversity, telecommunications and natural disaster management.

The Trace Transform method can be used to process these images. This method extracts features from the images by applying a series of functionals to produce a numeric representation that will be used for classification later on. This model depends on several factors in order to have an efficient operation, among them, the frequency parameters of the traces, classifier type and land coverage.

The purpose of this research is to create a model of sensibility to the frequency parameters of the Trace Transform for the classification of textures in aerial images.

Experimentation on feature extraction time and precision rate revealed that the frequency parameters, specially the pixel selection paramenter, $\Delta \tau$, and the classifier type have a big effect both of them.


\textbf{Keywords}: Trace transform, triple feature, image processing, extraction time, classification precision, aerial image.
\vspace*{\fill}



\singlespace