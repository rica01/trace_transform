\chapter{Objetivos y aportes}

\section{Objetivo general}

Estudiar el comportamiento de la TT en términos de precisión y tiempo de extracción de características, utilizando diferentes modelos de parámetros, clasificadores y tipos de terreno.

\section{Objetivos específicos}

\begin{enumerate}
    \item Comparar el tiempo de extracción de características y precisión de la clasificación de la TT utilizando diferentes combinaciones de los parámetros de frecuencia, cuatro distintos clasificadores y cinco tipos de cobertura de terreno.
    \item Analizar estadísticamente los resultados observados para determinar la influencia de los parámetros en el tiempo de extracción de características y precisión de la clasificación
\end{enumerate}
    
    
\section{Alcances y limitaciones}

Dentro del alcance de esta propuesta se definen los siguientes entregables:
\begin{enumerate}
    \item Prototipo de un programa para la extracción de características usando la TT. En este prototipo deben ser parametrizables, los parámetros de frecuencia y el clasificador.
    \item Prototipo de los clasificadores.
    \item Programas auxiliares para la ejecución y el control de los dos anteriores.
    \item Análisis estadístico para contrastar los resultados de los experimentos.
    \item Artículo científico que se entregará al comité editorial de alguna revista o conferencia, con miras a su publicación.
\end{enumerate}

Por motivos de tiempo y extensión, en esta investigación no se tomarán en cuenta:
\begin{itemize}
    \item Imágenes con formas irregulares, sólo se utilizarán imágenes rectangulares.
    \item Imágenes de tamaños variables. Se utilizarán imágenes en formato PNG, con un tamaño de $500 \times 500$ píxeles.
    \item Funcionales fuera de aquellas revisadas en la literatura. Se utilizarán los funcionales descritas en \cite{Petrou2007}.
    \item Una ventana de análisis en la imagen, ni deslizante ni de tamaño variable. La ventana de análisis será del tamaño total de la imagen.
\end{itemize}

Tampoco es la intención de esta investigación presentar un sistema completo para la clasificación de imágenes o para la generación de mapas de cobertura.

Por último, no se tomará en cuenta cualquier otro resultado, documento, software o producción que no esté contemplado en esta propuesta.
